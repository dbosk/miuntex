\documentclass[a4paper,protocol]{miunprot}

\usepackage[english,swedish]{babel}
\usepackage[T1]{fontenc}
\usepackage[utf8]{inputenc}
\usepackage{SIunits}

\groupname{APT IKS}
\chairman{Lotta Frisk}
\writer{Daniel Bosk}
\location{L408}
\date{8 oktober 2012}
\starttime{08:45}
\endtime{09:55}

\nextdate{22 oktober 2012}
\nexttime{08:45}
\nextlocation{L408}
\nextchair{Roger Olsson}
\nextwriter{Annika Berggren}

\begin{document}
%\begin{attending}
%	Not available.
%\end{attending}
%\begin{absent}
%	Not available.
%\end{absent}
\begin{adjuncts}
	Gunilla Candell (guncan)
\end{adjuncts}

\agendaitem[leiols]{Bemanning industriell ekonomi}
\noindent
Det är problem med bemanningen på industriell ekonomi.
På sikt blir det svårt att köra programmet som det görs idag, med en inriktning 
mot informatik då ingen kan ha ämnet.
Det är en kurs i P2 som saknar lärare helt.
Problematik finns främst i P2, 3 och 4 för indek åk 4.

Satsning sker på datateknik och industriell ekonomi.
Om det behövs får det ske en anställning inom ämnet industriell ekonomi.
Ledningen var medvetna om problemet inför omställningen.

Osäkert med den kurs som Håkan ska hålla.
Finns en lösning om extern resurs i form av samarbete med Försäkringskassan.
I övrigt att byta ut någon kurs.


\agendaitem[stepet]{Vägglöss i L213}
\noindent
Datorsal L213 var stängd under helgen 5-6 oktober på grund av vägglöss.
Anticimex har behandlat salen och den är åter öppen.


\agendaitem[patost]{Sommaruniversitetet}
\noindent
Sommaruniversitetet genomfördes i somars.
Det är ett projekt för att öka internationaliseringen, med deltagare från olika 
universitet i Europa.
Förra temat var \emph{House of the future}, det generella temat är hållbar 
utveckling.

Det kommer att gå igen den 12-23 augusti 2013.
Om någon är intresserad av att delta eller har förslag på projekt kontakta 
Patrik.

Det är ett bredare projekt för att få in många olika kompetenser.
Deet ska vara av storleken 4-6 studenter som ska genomföra under två veckors 
tid.

Det ges som en kurs som kan ge \unit{5}{hp}.
Detta är valfritt, för att få poäng krävs att man skriver en rapport.
Tyvärr deltog enbart våra utbytesstudenter.


\agendaitem[guncan]{Studenter med funktionsnedsättning}
\noindent
Studenter med någon form av funktionsnedsättning kan få pedagogiskt stöd.
En student måste alltid uppvisa läkarintyg för funktionsnedsättningen.
Det inleds med att studenten kontaktar och samtalar med en av samordnarna, 
bland annat om vilket stöd de haft i tidigare haft.
De vill även testa olika typer av stöd för att se vilket som passar bäst.

Det är flest dyslektiker, men även förkommande neuropsykiatriska 
funktionshinder som ADHD och Aspergers syndrom.

För studenter med neuropsykiatriska funktionshinder består stödet oftast av en 
mentor som hjälper dem att organisera hur de ska ta itu med studierna, 
exempelvis att planera sin tid.

Dyslektiker får oftast stöd i form av längre skrivtid, att skriva tentamen 
i enskilt rum eller med hjälp av dator.
Vissa får låna inspelningsapparatur för att spela in föreläsningar istället för 
att anteckna, men de måste prata med föreläsare innan inspelning.
Det förkommer även att ersätta tentamen med muntlig tentamen eller uppdelad 
skrivtid vid koncentrationssvårigheter som ADHD.

Att poängtera är att dessa studenter ska nå samma lärandemål som övriga 
studenter.
Det är mätmetoderna som ska anpassas inte gränsvärdena för mätningen eller vad 
som mäts.
Det är examinator som beslutar om examinationsmetoder eftersom att alla metoder 
inte är tillämpliga i varje fall.

Vissa studenter vill, medan andra inte vill, att lärare och programansvariga 
ska veta om problematiken.
Det är ett känsligt ämne för vissa, de vill inte skilja sig från övriga 
studenter.

Den enda resurs som finns för detta är pengar.

Förra året hade 300 aktiva studenter pedagogiskt stöd.


\other
\otheritem[katlin]{Ämnesföreträdare forskarutbildningsämnet data- och 
systemvetenskap}
\noindent
Katarina föreslår en tillförordnad ämnesföreträdare eftersom att rollen kommer 
att förändras genom kommande beslut från fakultetsstyrelsen.
Slutsatsen är en tillfällig ämnesföreträdare tills vidare, och när 
fakultetsbeslutet genomförts kan man byta företrädare om det anses mest 
lämpligt.


\makenextmeeting
\end{document}
